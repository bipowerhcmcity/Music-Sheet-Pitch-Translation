\documentclass[a4paper,12pt]{report}

\usepackage[utf8]{inputenc}
% \usepackage[Vietnamese]{babel}
\usepackage{titling}
\usepackage{rotating}
%package the indent the first line in latex
\usepackage{indentfirst}
\usepackage{graphicx}
\usepackage{ragged2e}
\usepackage{ragged2e}
\usepackage{afterpage}
\usepackage{amsmath}
\usepackage{subcaption}
\usepackage{float}
\usepackage{alltt}
\usepackage{listings}
%adding bibliography
\usepackage[backend=biber]{biblatex}
\addbibresource{/home/pj/MEGA/ComVi/paper_writing/Mendeley/bibitex/library.bib}

\graphicspath{{/home/pj/Documents/proposal/}}
% \graphicspath{{~/coding/python/Music-Sheet-Pitch-Translation/Latex_paper/draft}}

%making the numbering in Roman format 
\renewcommand\thesection{\Roman{section}}
\renewcommand\thesubsection{\roman{subsection}}

\setlength{\droptitle}{-8cm}

\usepackage{tikz}

\pretitle{
    \begin{tikzpicture}[remember picture,overlay]
    \node[anchor=north west,yshift=-1.5pt,xshift=1pt]%
        at (current page.north west)
        {\includegraphics{vgu_logo}};
    \end{tikzpicture}
}
\posttitle{}

\begin{titlepage}
	\title{ MUSIC SHEET UNDERSTANDING AND TONES TRANSPOSITION}
	\author{}
\end{titlepage}


\begin{document}

\afterpage{\null\newpage}

\maketitle

\tableofcontents

\clearpage

\section{Research team members}
\begin{itemize}
	\item Team Leader:      \hfill Truong Minh Khoa
	\item Team member: 		\hfill Dinh Cong Minh
	\item Team member:		\hfill Nguyen Tho Anh Khoa
	\item Team member:	    \hfill Huynh Minh Triet
\end{itemize}


\section{Disclaimer} 
This report is a product of our team's work, unless otherwise referenced. In
addition, all opinions, results, conclusions, and recommendations are of our own
and may not represent the policies or opinions of the Vietnamese-German
University's Department of Engineering or the University as a whole. 

\clearpage

\section{Abstract} 

The field of Optical Music Recognition (OMR), a subfield in Artificial
Intelligence, is aimed at automating the translation or understanding of music
sheets \cite{Calvo-Zaragoza}.  However, there is a lack of applications to solve
the problem of musical tones transposition (the process of moving a collection
of notes up or down in pitch by a constant interval).  Such a problem in tones
transposition is highly labor-intensive if the musician has to reorder the notes
manually, often impossible if done during a performance. Our work proposes a way
in which musicians can perform tones transposition by scanning a music sheet,
input the number of shift tones or semitones required, and then the programm
will output an audio file or a new music sheet with all of the notes shifted to
the required pitch. 

% \section{Team's workload}
% Due to the nature of our work is in pure code I believe the best way to show how 
% much each member contribute to the project can be directly viewed via our github
% commit page:



\section{Introduction}

The topic of recognizing musical sheets, i.e., Optical Music Recognition (OMR),
is not a novel field of research. The term OMR first appeared in a paper written
by MIT scientists in the 60s.  During the last three decades until now, OMR is
an ever increasingly developing field and is capable of solving many problems
that involve music \cite{Shatri2020a}\\

More specifically, the current OMR systems of today are capable enough to
recognize a printed musical sheet and digitize it. The resulting output could be
a .midi file, or other types of sound files such as .way, .mp3. The vast
majority of those researches are dedicated to the common user, even for users
who are not educated on musical theory, but there is still a lack of products
that can be used by professional or enthusiast musicians. In reality, a common
problem that is encountered is the transposition of music tones, i.e., shifting up or down a constant interval of
semitones or tones for the whole music sheet. Currently to obtain a music
sheet with a few tones higher or lower the musician has to manually retype the
entire musical sheet by hand, which is labor-intensive and time-consuming.


\subsection{Our solution}

We propose an algorithm that can take in a pdf file or scanned music sheet, then
translate the written note either by hand or computer drawn into scientific
pitch notation \cite{Wikipedia2013}.  At which stage, the program can play the
song or shift the song's tones or semitones according to the musicians' needs.

\clearpage

\section{Proposed Method}
The process includes: first removing the lines on each staff for ease of musical
note detection. The second stage is translating the detected note and do note
recognition to translate notes into scientific pitch notaion. The third stage is
where an additional python function will transpose the semitones or tones of the song to give
the final output. 

\subsection{Line Removal}
By using the method in \textcite{Gomez2017} the staff's line removal algorithm
will first grayscale the image then invert the color of the image so that the
lines are now white and the background is black. Finally, using a kernel of the
form: \\
$
\begin{pmatrix}
	0 & 0 & 0\\
	1 & 1 & 1\\
	0 & 0 & 0
\end{pmatrix}
$\\
then run with the dilate() function built-in OpenCV, the horizontal lines will
be expanded so that its width is increasingly larger, and any white pixels, i.e.,
notes, that does not belong to the horizontal line, will be flipped to 0 and
becomes the background thus eliminating all of the notes.

result:
\begin{figure}[H]
    \centering
    \begin{subfigure}[t]{0.5\textwidth}
        \centering
        \includegraphics[height=2.0in]{b4_dilate}
        \caption{music sheet before line removal}
    \end{subfigure}%
    ~ 
    \begin{subfigure}[t]{0.5\textwidth}
        \centering
        \includegraphics[height=2.0in]{after_dilate}
        \caption{music sheet after line removal}
    \end{subfigure}
\end{figure}

\clearpage
Finally, flipping the output image will result in a music sheet with lines
removed:
\begin{center}
  \makebox[\textwidth]{\includegraphics[width=\linewidth]{staffRemoval.jpg}}
\end{center}

\clearpage

\subsection{Note Translation}
To translate the note from its input as pictorial data to scientific pitch notation so that the
computer can process the music sheet, three steps are required. First by using 
\textcite{Rosebrock} method to eliminate overlapping note positions. The second is to
reorder the notes that are on the same staff lines into a list, for each staff line
there is a corresponding list containing the positions of each note on that staff.
Finally, converting the note positions into scientific pitch notation.

\subsubsection{Eliminating overlaping positions}
By running the template matching function built-in OpenCV on the output staff
line removed music sheet will obtain a list of position of notes of music sheet.
However, there is a problem with multiple note positions overlapping with each
other.\\

Inorder to handle this we use  \textcite{Rosebrock} method call Faster Non
Maximum Suppression so that each note will only have one position entry, avoiding
duplication in our list of note positions.\\

\subsubsection{Note reordering} 

Next musical notes that belong to the same staff
will be grouped into the same list,  for each staff line there is a
corresponding list containing its note.  To do this, from the position of the
first line(the bottom line of each staff) move down for a distance of half a
staff height and then create a second point which is from the current point up
to two staff heights \footnote{the reason for moving half of a staff height from
the first line down and then moving two staff height up  is to account for notes
that are on the staff and notes that are on the ledger line}. Any note's
vertical position that fell in the range from those previously mentioned points
will be considered as being in the same staff \\

\clearpage

In music theory, if two staves are connected by a curly bracket on the left
that means they must be played simultaneously.

E.g:\\ 
% \includegraphics{two_staff}
\begin{figure}[h]
\makebox[\textwidth]{\includegraphics[width=\linewidth]{two_staff}}
\caption{Two staves that need to be played simultaneously}
\label{fig:two staves}
\end{figure}


In the example above the staff above with the treble clef is called the main
staff and the staff below with the bass clef is called the sub staff. The
algorithm will now initialize two lists MAIN[] and SUB[] to store the two staves
respectively.\\

% \begin{alltt}
%     \normalfont
Now the algorithm will move simultaneously through both staves (main staff and sub staff) and check the
notes iteratively. For example, our first note MAIN[0] and SUB[0], if they
are vertically aligned, meaning that they need to be played simultaneously,
the algorithm will then move MAIN[0] and SUB[0] to MAIN\_RE\_ORDERED and
SUB\_RE\_ORDERD. By the word "move", the algorithm will cut the note
position from the original MAIN list and paste it into the
MAIN\_RE\_ORDERED, same goes for SUB\_RE\_ORDERED.\\
% \end{alltt}

However, since some music sheet has  minor errors in printing which will result in minor
misalignment of the note, meaning they are still supposed to be play
simultaneously but their horizontal (x-axis) position are not exactly the
same. The function ReoderedStaffs() has a threshold value of 5, meaning if
the two notes deviate from each other, either to the left or right, less than
5 pixels will still be considered as being played simultaneously.\\

There will arise a case, in which there is only one note on one of the staff, meaning
only one note need to be played at that moment, the tuple (0,0) will be added
into the staff that doesn't have a note as a filler note. In the case that any one of the
staff ran out of notes before the other staff, continue to move through both of the
staves like before, but the tuple (0,0) will be filled in as notes for the staff that ran out of notes first
Doing this assure that the two lists MAIN\_RE\_ORDERED and SUB\_RE\_ORDERED will
always have the same number of elements in their list.\\

\noindent This is the result of the first five notes of the two staves in figure number
\ref{fig:two staves}\\

\begin{figure}[H]
\begin{verbatim}
(216, 253), (242, 260), (270, 253), (298, 285), (325, 278)
(216, 412), (271, 368), (298, 381), (0, 0),     (353, 368)
\end{verbatim}
\caption{The fourth position has a (0,0) tuple because there is no corresponding fourth
note on the sub staff}
\end{figure}

\subsubsection{Digitalization of the notes}
With the list of notes's position, to be able to make note transposition possible
they need to be translateed scientific pitch notation.\\

To achieve this we have create four lists containing scientific pitch notaion.
Two list for the main staff and two for the sub staff(one for notes above the
first staff line and one for notes below the first staff line).\\

The two list for the main staff:
\begin{figure}[h]
\centering
\makebox[\textwidth]{['E5','D5','C5','B4','A4','G4','F4','E4','D4','C4','B3','A3','G3','F3','E3','D3','C3']}
\caption{notes below the first staff line of the main staff}
\vspace{\baselineskip}
\makebox[\textwidth]{['E5','F5','G5','A5','B5','C6','D6','E6','F6','G6']}
\caption{notes above the first staff line of the main staff}
\end{figure}

\vspace{\baselineskip}
The two list for the sub staff:
\begin{figure}[h]
\centering
\makebox[\textwidth]{['G3','F3','E3','D3','C3','B2','A2','G2','F2','E2','D2','C2','B1','A1','G1','F1','E1']}
\caption{notes below the first staff of the sub staff}
\vspace{\baselineskip}
\makebox[\textwidth]{['G3','A3','B3','C4','D4','E4','F4','G4','A4','B4']}
\caption{notes above the first staff line of the sub staff}
\end{figure}

Whenever a note is encountered on the staff (scanning from left to right), a
number will be generated which will be used as the index to get the note
notation. But since each staff has two lists which can be choosen from. This
problem is resolved by the following formula:\\

\[D = \frac{2.(note\_position[1] - headlines[i])}{d}\]

\noindent D is the ouput\\
note\_position[1] is the vertical position of the note\\
headlines[i] is the position of the first staff line of each staff\\
d is the distance between two staff lines in a staff \footnote{the 
full equation is: \(D = note\_position[1] -headlines[i] / d/2\)
with d/2 divided by two because each note is separated with just half of the distance between
each staff line}\\

If D $<$ 0 the algorithm will use the list for notes below the first staff line
and D $\geq$ 0 will use the list of notes that are above the first staff line.\\

\[ I = \begin{cases} \mbox{D,} & \mbox{if } D \geq 0 \\ \mbox{-D,} &
\mbox{otherwise} \end{cases}\]

Continue to repeat this staff by staff for all staves in the music sheet will
result in a list of scientific pitch notations for each note in the music sheet
organized in chronological order.\\

\subsection{Tones transposition}
After the previous process the output is a list of scientific pitch notaion,
e.g., C3, B2, C3, E2, F2, E3, D3, C3.\\

In music theory shifting a musical note up or down a tone or semitone follow a
predictable pattern:

\begin{figure}[h]
    \makebox[\textwidth]{\includegraphics[width=\linewidth]{music_tones}}
    \caption{layout of a piano keyboard}
    \label{piano_keyboard}
\end{figure}

According to the graph above, the music note C3 shifted up a tone would become
C\#3, same goes for all other notes, they become a sharp note. Except for the note
E and B which when shifted up a semitone becomes F and C respectively.\\

To solve this problem, we create a list of scientific pitch notation arrange in
increasing order similar to figure \ref{piano_keyboard}.\\

\begin{figure}[h]
\noindent\fbox{%
    \parbox{\textwidth}{%

notes\_height= ['E1', 'F1','G1','A1', 'B1', 'C2','D2', 'E2', 'F2','G2','A2',
'B2', 'C3', 'D3', 'E3', 'F3','G3', 'A3', 'B3', 'C4', 'D4', 'E4', 'F4',
'G4','A4','B4', 'C5', 'D5', 'E5', 'F5', 'G5', 'A5', 'B5', 'C6', 'D6', 'E6',
'F6','G6','A6','B6' ,'C7','D7', 'E7', 'F7','G7','A7', 'B2',]
    }%
}
\caption{list of notes pitch in ascending order}
\label{list: note pitch increase}
\end{figure}


\medskip

In the list above in figure \ref{list: note pitch increase}, each note will have
a note that is one semitone higher on the left, e.g., E1 and F1.  Therefore to
shift a note tone up a semitone, first the note's position on the list, i.e.,
its index, must be determined; Second, the new shifted note is at the index of
the old note plus one.\\

A for loop is used to loop over the entire list and check each
iteration whether the current iteration matches, if yes return the index of the note.\\

\begin{lstlisting}[language=Python]
def findNoteinNote_height(note):
    for i in range(len(notes_height)):
        if(note == notes_height[i]):
            return i
\end{lstlisting}
\medskip

Once obtained the index of the note, to shift the tone's note up a semitone 
another for loop is used:\\

\begin{lstlisting}[language=Python]
if(chord.name!="E" and chord.name!="B" and chord.sharp==False):
    chord.sharp = True
else:
    chord.sharp = False
    chord.name +=1  # E + 1 = F, F+1 = G,etc
\end{lstlisting}
\medskip
The variable "chord.name" is the index of the new note in the list in figure
\ref{list: note pitch increase}.  Each time the loop above is run it will
increase all note on the staff by one semitone, so if the end-user wants to shift
the music sheet by three semitones, the for loop above will be run three
times.\\

The same process is used to shift the music notes down but instead of increasing
"chord.name" by one, "chord.name" will be decreased by one. In addition, there are
special cases for when the note is either F or C when shifting the music note
down a semitone just like the cases B and C above.


\section{Future work}
With the help of Nguyen Tho Anh Khoa we are
planning to use differential binarization for musical note detection and CRNN
for note recognition. With these methods in text recognition, we are hoping to
build our new version to be more robust and capable of handling handwritten music
sheets, or music sheets that have staff lines not perfectly straights or
equally parallel.\\

Future version will implement the model in
\textcite{Pacha2017} to bring this method to android devices.

\section{Expected outcome}
Since tone transposition is a common problem faced by many musicians, our
solution must logically also be widely available and easy to use. For this
reason, our final goal is to implement the entire method as a mobile app that
can be download and installed from app stores and use immediately.  


\clearpage

\section{Timeline}


\newcommand\measurepage{\dimexpr\pagegoal-\pagetotal-\baselineskip\relax}
\begin{sidewaysfigure}[hbt!]
    \includegraphics[width = \textwidth, height = \measurepage, keepaspectratio]{Schedule.png}
\end{sidewaysfigure}

% \begin{figure}[h]
%     % \makebox[\linewidth]{\includegraphics[width=\linewidth]{Schedule.png}}
%     % \includegraphics{Schedule.png}
%     \includegraphics[width = \linewidth, height = \textheight,keepaspectratio]{Schedule.png}
%     \caption{projected timeline in the scale of one year}
%     \label{Timeline}
% \end{figure}



Task description:\\

Detecting staff line: staff line detection is the first essential step to obtain
the coordinates of each line from which the note's pitch can be derived.\\

Staff line removal: Run Length Encoding or other methods are used to remove
every line of a line-detected music sheet from the previous process or an
independent music sheet; this is done for ease of symbols and note recognition.\\

Note and note type detection: At this stage, template matching is then used to
recognize the shape of each music note as well as categorize each note to
determine whether it's a whole note, half note, or a quarter note, etc.\\

Grouping Note to Staff: This process will arrange each note into its respective
staff. From then on, operation on the music sheet is not done on the entire sheet
but is done staff by staff; this is a divide and conquers method\\

Detecting additional symbols: A music sheet doesn't just contain music notes, but
it might also include accidentals and key signatures to indicate beat and tone.
Recognition and understanding of the semantic of each symbol enable the checking
of whether the staff has the number of beats required.\\

Note Translation: At this stage, each note will be converted to
scientific pitch notation, e.g., A, B and C, etc. based on its positional
coordinate.\\

Note Transposition: This process shifts the note up or down based on its
scientific pitch notation to result in a transposed music sheet.\\

Build an app to display the result: at this stage, the final product, i.e., the
transposed music sheet is completed; we will build an app to apply and display
the final result on mobile devices.\\

Write paper: During the last two months of the project, we will start the
process of writing a scientific paper; as a way to contribute to the scientific
community and concluding a year of work.\\


\printbibliography

\end{document}