\documentclass[a4paper,12pt]{report}

\usepackage[utf8]{inputenc}
% \usepackage[Vietnamese]{babel}
\usepackage{titling}
%package the indent the first line in latex
\usepackage{indentfirst}
\usepackage{graphicx}
\usepackage{ragged2e}
\usepackage{ragged2e}
\usepackage{afterpage}
\usepackage{amsmath}

%adding bibliography
\usepackage[backend=biber]{biblatex}
\addbibresource{/home/pj/MEGA/ComVi/paper_writing/Mendeley/bibitex/library.bib}

\graphicspath{{/home/pj/Documents/proposal/}}
% \graphicspath{{~/coding/python/Music-Sheet-Pitch-Translation/Latex_paper/draft}}

%making the numbering in Roman format 
\renewcommand\thesection{\Roman{section}}
\renewcommand\thesubsection{\roman{subsection}}

\setlength{\droptitle}{-8cm}

\usepackage{tikz}

\pretitle{
    \begin{tikzpicture}[remember picture,overlay]
    \node[anchor=north west,yshift=-1.5pt,xshift=1pt]%
        at (current page.north west)
        {\includegraphics{vgu_logo}};
    \end{tikzpicture}
}
\posttitle{}

\begin{titlepage}
	\title{ MUSIC SHEET UNDERSTANDING AND TONE MODULATION}
	\author{}
\end{titlepage}


\begin{document}

\afterpage{\null\newpage}

\maketitle

\tableofcontents

\clearpage

\section{Research team members}
\begin{itemize}
	\item Team Leader:      \hfill Truong Minh Khoa
	\item Programmer: 		\hfill Dinh Cong Minh
	\item Programmer:		\hfill Nguyen Tho Anh Khoa
	\item Writer/Editor:	\hfill Huynh Minh Triet
\end{itemize}


\section{Disclaimer} 
This report is a product of our team's work, unless otherwise referenced. In
addition, all opinions, results, conclusions, and recommendations are of our own
and may not represent the policies or opinions of the Vietnamese-German
University's Department of Engineering or the University as a whole. 

\clearpage

\section{Abstract}

\section{Introduction}

The topic of recognizing musical sheets, i.e., Optical Music Recognition (OMR),
is not a novel field of research. The term OMR first appeared in a paper written
by MIT scientists in the 60s.  During the last three decades until now, OMR is
an ever increasingly developing field and is capable of solving many problems
that involves with music

More specifically, the current OMR systems of today are capable enough to
recognize a printed musical sheet and digitize it. The resulting output could be
a .midi file, or other types of sound files such as .way, .mp3. The vast
majority of those researches are dedicated for the common user, even for users
who are not educated on musical theory, but there is still a lack of product
that can be used for professional or enthusiast musicians. In reality, a common
problem that is encountered is the modulation of music tones, i.e., up or down
semitone, tone for the whole music sheet. Currently in order to obtain a music
sheet with a few tones higher or lower the musicsian has to manually retype the
entire musical sheet by hand, which is labor intensive.

\subsection{Our solution}
We propose an algorithm that can take in a pdf file or scanned music sheet, then 
translate the written note either by hand or computer drawn into a digital format.
At which stage, the program can play the song or shift the song's tones or semitomes 
according to the musicians' need.

\clearpage

\section{Proposed Method}
The process include first removing the lines on each staff for ease of musical
note dectection.  The second stage is translating the detected note and do note
recognition to translate notes into a digital format. Finnaly the last stage is
where an additional python function will modulate the tones of the song to give
the final output. 


\subsection{Line Removal}
By using the method in \textcite{Gomez2017} the staff's line removal algorithm
will first grayscale the image then invert the color of the image, so that the
lines are now white and the background is black. Finally, using a kernel of the
form: \\
$
\begin{pmatrix}
	0 & 0 & 0\\
	1 & 1 & 1\\
	0 & 0 & 0
\end{pmatrix}
$\\
then run with the dilate() function built-in OpenCV, the
horizontal lines will be expand so that its width is increasingly larger, and
any white pixels, i.e., notes, that does not belong to the horizontal line, will
be flipped to 0 and becomes the background thus eliminating notes.

for example:

\begin{figure}
    \centering
    \begin{minipage}{0.45\textwidth}
        \centering
        \includegraphics{b4_dilate} % first figure itself
        \caption{before note removal}
    \end{minipage}\hfill
    \begin{minipage}{0.45\textwidth}
        \centering
        \includegraphics{after_dilate} % second figure itself
        \caption{after note removal}
    \end{minipage}
\end{figure}

\printbibliography


\end{document}
