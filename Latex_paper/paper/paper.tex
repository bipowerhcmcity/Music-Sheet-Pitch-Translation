%Paper todo list
% - Write note grouping
% - Multiscale part
% - Rewrite Abstract (include more method in it )
% - rewrite tone transposition

\documentclass[final]{cvpr}

\usepackage{times}
\usepackage{epsfig}
\usepackage{graphicx}
\usepackage{amsmath}
\usepackage{amssymb}

% other packages
\graphicspath{{/home/pj/Documents/proposal/}} %images 

\usepackage[pagebackref=true,breaklinks=true,colorlinks,bookmarks=false]{hyperref} %referencing 

\usepackage[utf8]{inputenc}
\usepackage{titling}
\usepackage{rotating}
%package the indent the first line in latex
\usepackage{indentfirst}
\usepackage{ragged2e}
\usepackage{afterpage}
\usepackage{subcaption}
\usepackage{float}
\usepackage{alltt}
\usepackage{listings}




\begin{document}

%%%%%%%%% TITLE
\title{Music Sheet Understanding and Tones Transposition}

\author{Khoa Truong,Minh Dinh,Triet Huynh,and Dr.Vinh Dinh\\
         CSE,Vietnamese-German University,Thu Dau Mot City
\and
Khoa Nguyen\\
CS,alumni of HCM University of Technology K14, HCM City
}  

% \author{Khoa Truong,Minh Dinh,Triet Huynh,Khoa Nguyen and Vinh Dinh\\
%          CSE,Vietnamese-German University}  

\maketitle


%%%%%%%%% ABSTRACT
\begin{abstract}
The field of Optical Music Recognition (OMR), a subfield in Artificial
Intelligence, is aimed at automating the translation or understanding of music
sheets \cite{Calvo-Zaragoza}.  However, there is a lack of applications to solve
the problem of musical tones transposition (the process of moving a collection
of notes up or down in pitch by a constant interval).  Such a problem in tones
transposition is highly labor-intensive if the musician has to reorder the notes
manually, often impossible if done during a performance. Our work proposes a way
in which musicians can perform tones transposition by scanning a music sheet,
input the number of shift tones or semitones required, and then the programm
will output an audio file or a new music sheet with all of the notes shifted to
the required pitch. 
\end{abstract}

%%%%%%%%% BODY TEXT
\section{Introduction}
The topic of recognizing musical sheets, i.e., Optical Music Recognition (OMR),
is not a novel field of research. The term OMR first appeared in a paper written
by MIT scientists in the 60s.  During the last three decades until now, OMR is
an ever increasingly developing field and is capable of solving many music related
problems \cite{Shatri2020a}.

More specifically, the current OMR systems of today are capable enough to
recognize a printed musical sheet and digitize it. The resulting output could be
a .midi file, or other types of sound files such as .wav, .mp3. The vast
majority of those researches are dedicated to the common user, even for users
who are not educated on musical theory, but there is still a lack of products
that can be used by professional or enthusiast musicians. In reality, a common
problem that is encountered is the transposition of music tones, i.e., shifting up or down a constant interval of
semitones or tones for the whole music sheet. Currently to obtain a music
sheet with a few tones higher or lower the musician has to manually retype the
entire musical sheet by hand, which is labor-intensive and time-consuming.

%-------------------------------------------------------------------------
\section{Proposed Method}
The process includes the first stage, Staff Line Preprocessing, to remove the
lines on each staff for ease of musical note detection. The second stage is Note
Translation to translates the detected notes into scientific pitch notation. The
third stage, Tone Transposition will transpose the semitones or tones of the
song to give the final output. 

\subsection{Staff Line Preprocessing}
For ease of note recognition, first, the staff lines must be detected and
removed. This process includes two stages. First, detection of the staff lines
and their thickness; second, removal of those staff lines.

\subsubsection{Staff Line Detection}
To detect the staff line in the music sheet we first need to remove all of the notes.
This is done for ease of staff line detection.

Our staff's line detection algorithm will first grayscale the image and then
invert the image colors so that the lines are now white and the background is
black. The color inverted music sheet is then run with the OpenCV built-in
dilate() function, where the horizontal lines will be expanded so that its width
is increasingly larger, and any white pixels, i.e., notes, that does not belong
to the horizontal line, will be flipped to 0 and becomes the background thus
eliminating all of the notes.
\clearpage
Result:
\begin{figure}[H]
    \centering
    \begin{subfigure}[t]{0.2\textwidth}
        \centering
        \includegraphics[height=1.5in]{b4_dilate}
        \caption{before applying dilate function}
    \end{subfigure}%
    ~ 
    \begin{subfigure}[t]{0.2\textwidth}
        \centering
        \includegraphics[height=1.5in]{after_dilate}
        \caption{after applying dilate function}
    \end{subfigure}
\end{figure}

the dilated image above will give an output that is fairly accurate except for a
few lines at the last staff line. this is why we then use Gomez's
method \cite{Gomez2017} and run length encoding (RLE) to determine the lines' thicknesses and
their distance between one another. RLE will be applied vertically along a
random column of the dilated image above:

\begin{figure}[ht]
\makebox[\columnwidth]{\includegraphics[scale = 0.5]{apply_RLE}}
\caption{This image was taken from Gomez's original paper \cite{Gomez2017} which was done on a
note removed but not color inverted music sheet. which is why the line is in
black and the distance is in white, opposite to our dilated image.}
\end{figure}


RLE will group adjacent pixels of the same color into one group with a number to
indicate how many pixels of the same color is within that group and a character
to indicate that group is either a group of black pixels "b" or a group of white
pixels "w". through which the staff line thickness and the distance between two
staff lines are then obtained. more specifically, to extract the distance between each
staff line, e.g., our RLE output might be 3b,2w,4b,2w,3b,3w,3b,2w, the black
group (because our music sheet is color inverted so black is the distance) that
appear most often "3b" with the value of three pixels is chosen. therefore, the
distance between two staff lines is three pixels. the same method is also applied
to get the thickness of the lines.

\subsubsection{Staff Line Removal}
In our music sheet, places where there is only lines will have matrix of 
this type:

\begin{figure}[ht]
    \centering
    $
\begin{pmatrix}
	0 & 0 & 0\\
	1 & 1 & 1\\
	0 & 0 & 0
\end{pmatrix}
$\\
\caption{an example of a kernel with one line and no notes}
\label{line only matrix}
\end{figure}


Please note that the rows of ones represent the thickness of the line, which can
vary from one to multiple rows, the same thing goes for the row of zeros, distance
between each line.

\begin{figure}[h]
   %  \makebox[\columnwidth]{\includegraphics[width=\linewidth]{inverted_two_staves.png}}
    \makebox[\columnwidth]{\includegraphics[width=\linewidth]{inverted_two_staves.png}}
    \caption{The current color inverted music sheet}
    \label{}
\end{figure}
 
To remove the staff lines on each music line, a check vector of length N will be run 
on each NxN kernel of the image:

\begin{figure}[ht]
    \centering
$
\begin{pmatrix}
	1 & ... & 1 & 0 & ... & 0 & 1 & ... & 1
\end{pmatrix}
$\\
\caption{Check vector of length N}
\label{check vector}
\end{figure}

The number of zeros and ones of the N length check vector in figure \ref{check
vector} depends on the thickness of each line and the distance between two staff
lines respectively, which was obtained in the previous section.  For example,
the kernel with one line and no notes in figure \ref{line only matrix} the
check vector will be of the type:

\begin{figure}[h!]
    \centering
    $
    \begin{pmatrix}
       1 & 0 & 1 
    \end{pmatrix}
    $\\
\end{figure}

 This check vector is then multiplied with every NxN kernel of the image. If the
 kernel is of the type of a line, the result will be a vector of 0s, otherwise,
 the resulting vector will contain at least one positive integer among its
 elements.

 For every time we encounter an area that is a line by using the check vector,
 the found area will be replaced by the NxN matrix which is full of 0s, else it
 will be ignored and we will continuously check the next kernel. The process
 will be terminated when we have checked all the possible kernels in the
 image.

Finally, inverting the color of the output image one last time will result in a
music sheet with lines removed:

\makebox[\columnwidth]{\includegraphics[width=\linewidth]{staffRemoval.jpg}}

\subsection{Note Translation}

To translate the note from its input as pictorial data to scientific pitch
notation so that the computer can process the music sheet, three steps are
required. First by using Rosebrock's method \cite{Rosebrock} to eliminate overlapping
note positions after running template matching. The second is to group the
notes that are on the same staff lines into a list, for each staff line there is
a corresponding list containing the positions of each note on that staff.
Finally, converting the note positions into scientific pitch notation.

\subsubsection{Eliminating overlaping positions}
By running the template matching function built-in OpenCV on the output staff
line removed music sheet will obtain a list of position of notes of music sheet.
However, there is a problem with multiple note positions overlapping with each
other.

Inorder to handle this we use Rosebock's method \cite{Rosebrock} call Faster Non
Maximum Suppression so that each note will only have one position entry, avoiding
duplication in our list of note positions.

\subsubsection{Note reordering} 

% Next musical notes that belong to the same staff
% will be grouped into the same list,  for each staff line there is a
% corresponding list containing its note.  To do this, from the position of the
% first line(the bottom line of each staff) move down for a distance of half a
% staff height and then create a second point which is from the current point up
% to two staff heights (reason for moving half of a staff height from
% the first line down and then moving two staff height up  is to account for notes
% that are on the staff's lines and notes that are on the ledger lines). Any notes'
% vertical position that fell in the range from those previously mentioned points
% will be considered as being in the same staff. 


In music theory, if two staves are connected by a curly bracket on the left
that means they must be played simultaneously.

E.g: 
% \includegraphics{two_staff}
\begin{figure}[h]
\makebox[\columnwidth]{\includegraphics[width=\linewidth]{two_staff}}
\caption{Two staves that need to be played simultaneously}
\label{fig:two staves}
\end{figure}


In the example above the staff above with the treble clef is called the main
staff and the staff below with the bass clef is called the sub staff. The
algorithm will now initialize two lists MAIN[] and SUB[] to store the two staves
respectively.

% \begin{alltt}
%     \normalfont
Now the algorithm will move simultaneously through both staves (main staff and sub staff) and check the
notes iteratively. For example, our first note MAIN[0] and SUB[0], if they
are vertically aligned, meaning that they need to be played simultaneously,
the algorithm will then move MAIN[0] and SUB[0] to MAIN\_RE\_ORDERED and
SUB\_RE\_ORDERD. By the word "move", the algorithm will cut the note
position from the original MAIN list and paste it into the
MAIN\_RE\_ORDERED, same goes for SUB\_RE\_ORDERED.\\
% \end{alltt}

However, since some music sheet has  minor errors during printing which will result in minor
misalignment of the note, meaning they are still supposed to be play
simultaneously but their horizontal (x-axis) position are not exactly the
same. The function ReoderedStaffs() has a threshold value of 5, meaning if
the two notes deviate from each other, either to the left or right, less than
5 pixels will still be considered as being played simultaneously.\\

There will arise a case, in which there is only one note on one of the staff, meaning
only one note is needed to be played at that moment, the tuple (0,0) will be added
into the staff that doesn't have a note as a filler note. In the case that any one of the
staff ran out of notes before the other staff, continue to move through both of the
staves like before, but the tuple (0,0) will be filled in as notes for the staff that ran out of notes first.
Doing this will assure that the two lists MAIN\_RE\_ORDERED and SUB\_RE\_ORDERED will
always have the same number of elements in their list.\\

\noindent This is the result of the first five notes of the two staves in figure 
where there are two staves that need to be played simultaneously.
\ref{fig:two staves}:\\

\noindent(216, 253), (242, 260), (270, 253), (298, 285), (325, 278)
         (216, 412), (271, 368), (298, 381), (0, 0), \hfill    (353, 368)


\subsubsection{Digitalization of the notes}
With the list of notes's position, to be able to make note transposition possible
they need to be translateed into scientific pitch notation.\\

To achieve this we have create four lists containing scientific pitch notaion.
Two list for the main staff and two for the sub staff(one for notes above the
first staff line and one for notes below the first staff line).\\

The two list for the main staff:
\begin{figure}[h]
% \makebox[\columnwidth]{['E5','D5','C5','B4','A4','G4','F4','E4','D4','C4','B3','A3','G3','F3','E3','D3','C3']}
['E5','D5','C5','B4','A4','G4','F4','E4','D4','C4',' B3','A3,
'G3','F3','E3','D3','C3']
\caption{notes below the first staff line of the main staff}
\vspace{\baselineskip}
\makebox[\columnwidth]{['E5','F5','G5','A5','B5','C6','D6','E6','F6','G6']}
\caption{notes above the first staff line of the main staff}
\end{figure}

\vspace{\baselineskip}
The two list for the sub staff:
\begin{figure}[h]
% \makebox[\columnwidth]{['G3','F3','E3','D3','C3','B2','A2','G2','F2','E2','D2','C2','B1','A1','G1','F1','E1']}
['G3','F3','E3','D3','C3','B2','A2','G2','F2','E2','D2,
'C2', 'B1','A1','G1','F1','E1']
\caption{notes below the first staff of the sub staff}
\vspace{\baselineskip}
\makebox[\columnwidth]{['G3','A3','B3','C4','D4','E4','F4','G4','A4','B4']}
\caption{notes above the first staff line of the sub staff}
\end{figure}

Whenever a note is encountered on the staff (scanning from left to right), a
number will be generated which will be used as the index to get the note
notation. But since each staff has two lists which can be choosen from. This
problem is resolved by the following formula:\\

\[D = \frac{2.(note\_position[1] - headlines[i])}{d}\]

\noindent D is the ouput\\
note\_position[1] is the vertical position of the note\\
headlines[i] is the position of the first staff line of each staff\\
d is the distance between two staff lines in a staff \footnote{the 
full equation is: \(D = note\_position[1] -headlines[i] / d/2\)
with d/2 divided by two because each note is separated with just half of the distance between
each staff line}\\

If D $<$ 0 the algorithm will use the list for notes below the first staff line
and D $\geq$ 0 will use the list of notes that are above the first staff line.\\

\[ I = \begin{cases} \mbox{D,} & \mbox{if } D \geq 0 \\ \mbox{-D,} &
\mbox{otherwise} \end{cases}\]

Continue to repeat this staff by staff for all staves in the music sheet will
result in a list of scientific pitch notations for each note in the music sheet
organized in chronological order.

\subsection{Tones Transposition}
After the previous process the output is a list of scientific pitch notaion,
e.g., C3, B2, C3, E2, F2, E3, D3, C3.

In music theory shifting a musical note up or down a tone or semitone follow a
predictable pattern:

\begin{figure}[h]
    \makebox[\columnwidth]{\includegraphics[width=\linewidth]{music_tones}}
    \caption{layout of a piano keyboard}
    \label{piano_keyboard}
\end{figure}

According to the graph above, the music note C3 shifted up a tone would become
C\#3, same goes for all other notes, they become a sharp note. Except for the note
E and B which when shifted up a semitone becomes F and C respectively.\\

To solve this problem, we create a list of scientific pitch notation arrange in
increasing order similar to figure \ref{piano_keyboard}.\\

\begin{figure}[H]
\noindent\fbox{%
    \parbox{\columnwidth}{%

notes\_height= {'E1':0,'F1':1,'G1':2,'A1':3,'B1':4,'C2':5,'D2':6,
'E2':7,'F2':8,'G2':9,'A2':10,'B2':11, 'C3':12, 'D3':13,
'E3':14,'F3':15,'G3':16,'A3':17, 'B3':18, 'C4':19, 'D4':20, 'E4':21, 'F4':22,
'G4':23,'A4':24,'B4':25, 'C5':26, 'D5':27, 'E5':28, 'F5':29,'G5':30, 'A5':31,
'B5':32, 'C6':33, 'D6':34, 'E6':35, 'F6':36,'G6':37,'A6':38,'B6':39
,'C7':40,'D7':41, 'E7':42, 'F7':43,'G7':44,'A7':45, 'B7':46}   }%
}
\caption{list of notes pitch in ascending order}
\label{list: note pitch increase}
\end{figure}

% In the dictionary list above in figure \ref{list: note pitch increase}, each
% note will have a note that is one semitone higher on its the left, e.g., E1 and F1.
% Therefore to shift a note tone up a semitone the note's position in the list,
% i.e., its index, must be determined; then the new shifted note is at the index
% of the old note plus one.

In the the list above in \ref{list: note pitch increase} which is built using python
dictionary, Wherein each dictionary items has a key (the scientific pitch
notaion),e.g.,E1,F1,etc and a value (the index),0,1,etc respectively.each
note will have a note that is one semitone higher on its the left, e.g., E1 and F1.
% Therefore to shift a note tone up a semitone the note's position in the list,
% i.e., its index, must be determined, this is done by using the following function

\begin{lstlisting}[language=Python]
if(chord[3] == False and str(chord[2][0])\
!='E' and str(chord[2])[0] !='B' ):
chord[3] = True
else:
chord[3] = False
index = notes_height[chord[2]]
new_index = int(index) + 1
chord[2]=list(notes_height.keys())\
[new_index]
\end{lstlisting}

Once obtained the index of the note, to shift the tone's note up a semitone 
another for loop is used:\\

% \begin{lstlisting}[language=Python]
% if(chord.name!="E" and chord.name!="B" and 
% chord.sharp==False):
%     chord.sharp = True
% else:
%     chord.sharp = False
%     chord.name +=1  # E + 1 = F, F+1 = G,etc
% \end{lstlisting}

The variable "chord.name" is the index of the new note in the list in figure
\ref{list: note pitch increase}.  Each time the loop above is run it will
increase a note on the current staff by one semitone, so if the end-user wants to shift
the music sheet by three semitones, the for loop above will be run three
times for each note.\\

The same process is used to shift the music notes down but instead of increasing
"chord.name" by one, "chord.name" will be decreased by one. In addition, there are
special cases for when the note is either F or C when shifting the music note
down a semitone just like the cases B and C above.

\section{Future work}
With the help of Nguyen Tho Anh Khoa we are
planning to use differential binarization for musical note detection and CRNN
for note recognition. With these methods in text recognition, we are hoping to
build our new version to be more robust and capable of handling badly handwritten music
sheets, or music sheets that have staff lines not perfectly straights or
equally parallel.\\


\section{Expected outcome}
Our outcome includes a demo source code that can demonstrate tone transposition and at least one conference
paper. 




{\small
\bibliographystyle{ieee_fullname}
\bibliography{/home/pj/MEGA/ComVi/paper_writing/Mendeley/bibitex/library.bib}
}

\end{document}
